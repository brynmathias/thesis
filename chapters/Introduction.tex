\chapter{Introduction} % (fold)
\label{cha:introduction}
The Large Hadron Collider (LHC) \cite{Benedikt:823808} is a proton-proton collider which is situated in the Large Electorn Positron (LEP) tunnel under the
franco-swiss border. Design center of mass energy is 14 TeV with an instantanious luminosity of $ 1 \times 10$e$^{34}$
cm$^{-2}$s$^{-1}$. However duing 2011 the center of mass energy was 7 TeV and the maximum luminosity was $ 5 \times 10$e$^{33}$
cm$^{-2}$s$^{-1}$.
To achieve this high energy the LHC uses superconducting niobium-titanium magnets that produce a maximum field strength of 8.36
Tesla. The magents are cooled to a minimum temprature of 1.8 Kelvin.

Situated around the LHC ring are four detectors, two general detectors ATLAS \cite{Akesson:1999uv} and CMS
\cite{Friedl:1140134}\cite{Wulz:vf} designed to measure the standard model to high precision and search for new physics. The LHC beauty
experiment \cite{Rademacker:2005tx} is designed to study at previously unattainable precision the decays of heavy quark flavors, both to
measure the standard model couplings and to search for beyond the standard model (BSM) physicical processes. Finally the ALICE \cite{Alessandro:2006ht}
experiment is designed to run when the LHC is running in it's secondary mode where rather than proton bunches are collided, lead
ions are collided, in an effort to study the quark-gluon plasma.
% chapter introduction (end)